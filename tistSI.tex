% v2-acmsmall-sample.tex, dated March 6 2012
% This is a sample file for ACM small trim journals
%
% Compilation using 'acmsmall.cls' - version 1.3 (March 2012), Aptara Inc.
% (c) 2010 Association for Computing Machinery (ACM)
%
% Questions/Suggestions/Feedback should be addressed to => "acmtexsupport@aptaracorp.com".
% Users can also go through the FAQs available on the journal's submission webpage.
%
% Steps to compile: latex, bibtex, latex latex
%
% For tracking purposes => this is v1.3 - March 2012

\documentclass[prodmode,acmtecs]{acmsmall} % Aptara syntax

% Package to generate and customize Algorithm as per ACM style
\usepackage[ruled]{algorithm2e}
\renewcommand{\algorithmcfname}{ALGORITHM}


\usepackage{graphicx}
\usepackage{float}
\usepackage{color}
\usepackage{times}
\usepackage{helvet}
\usepackage{courier}
\usepackage[cmex10]{amsmath}
\usepackage{algcompatible}
\usepackage{multirow}
%\usepackage{amsthm}
\usepackage{mdwlist}
\usepackage{cite}
\usepackage[sort&compress]{natbib}
\usepackage{epsfig}
\usepackage{epstopdf}
\usepackage{url}
\usepackage{multirow}
\usepackage{amsmath}
\usepackage{mathrsfs}
\usepackage{enumerate}
\usepackage{comment}
%\usepackage{algorithmicx}
%\usepackage{algorithm} % http://ctan.org/pkg/algorithms
\usepackage{array}
\usepackage[nottoc]{tocbibind}
\newcommand{\argmin}{\arg\!\min}
\algrenewcommand\alglinenumber[1]{\tiny #1:}

%\newtheorem{definition}{Definition}

\SetAlFnt{\small}
\SetAlCapFnt{\small}
\SetAlCapNameFnt{\small}
\SetAlCapHSkip{0pt}
\IncMargin{-\parindent}

% Metadata Information
%\acmVolume{9}
%\acmNumber{4}
%\acmArticle{39}
%\acmYear{2010}
%\acmMonth{3}

% Document starts
\begin{document}

% Page heads
\markboth{H. Shao et al.}{Temporal Motif Mining Approaches for Smart Homes}

% Title portion
\title{Temporal Motif Mining Approaches for Smart Homes}
\author{Huijuan Shao
\affil{Virginia Tech}
Yaowei Li
\affil{Game source}
Fei Li
\affil{Virginia Tech}
Erin Griffiths
\affil{University of Virginia}
Kamin Whitehouse
\affil{University of Virginia}
Naren Ramakrishnan
\affil{Virginia Tech}}
% NOTE! Affiliations placed here should be for the institution where the
%       BULK of the research was done. If the author has gone to a new
%       institution, before publication, the (above) affiliation should NOT be changed.
%       The authors 'current' address may be given in the "Author's addresses:" block (below).
%       So for example, Mr. Abdelzaher, the bulk of the research was done at UIUC, and he is
%       currently affiliated with NASA.

\begin{abstract}
With the advent of modern sensor technologies, 
significant opportunities have opened up to help conserve energy in 
residential and commercial buildings. Moreover, the rapid \emph{urbanization} we are witnessing requires optimized energy distribution. 
This paper focuses on two sub-problems in improving energy conservation; \emph{energy disaggregation and occupancy prediction}. 
Energy disaggregation attempts to 
separate the energy usage 
of each circuit or each electric device in a building 
using only aggregate electricity usage information from 
the meter for the whole house. 
The second problem of occupancy prediction can be accomplished using non-invasive indoor activity tracking to 
predict the locations of people inside a building. 
We cast both problems as \emph{temporal mining problems}. We exploit motif mining with constraints to distinguish devices with multiple states, which helps tackle the energy disaggregation problem. Our
results reveal that motif mining is adept at distinguishing
devices with multiple power levels and at disentangling the
combinatorial operation of devices.
For the second problem we propose time-gap constrained episode mining to detect 
activity patterns followed by the use of a mixture of episode generating HMM (EGH) models 
to predict home occupancy.  
Finally, we demonstrate that the mixture EGH
model can also help predict the location of a person to 
address non-invasive indoor activities tracking. 
\end{abstract}

%\category{C.2.2}{Computer-Communication Networks}{Network Protocols}

%\terms{Design, Algorithms, Performance}

\keywords{Data mining, sustainability, energy disaggregation, occupancy prediction}

%\acmformat{Gang Zhou, Yafeng Wu, Ting Yan, Tian He, Chengdu Huang, John A. Stankovic,
%and Tarek F. Abdelzaher, 2010. A multifrequency MAC specially
%designed for  wireless sensor network applications.}
% At a minimum you need to supply the author names, year and a title.
% IMPORTANT:
% Full first names whenever they are known, surname last, followed by a period.
% In the case of two authors, 'and' is placed between them.
% In the case of three or more authors, the serial comma is used, that is, all author names
% except the last one but including the penultimate author's name are followed by a comma,
% and then 'and' is placed before the final author's name.
% If only first and middle initials are known, then each initial
% is followed by a period and they are separated by a space.
% The remaining information (journal title, volume, article number, date, etc.) is 'auto-generated'.

%\begin{bottomstuff}
%This work is supported by the National Science Foundation, under
%grant CNS-0435060, grant CCR-0325197 and grant EN-CS-0329609.

%Author's addresses: G. Zhou, Computer Science Department,
%College of William and Mary; Y. Wu  {and} J. A. Stankovic,
%Computer Science Department, University of Virginia; T. Yan,
%Eaton Innovation Center; T. He, Computer Science Department,
%University of Minnesota; C. Huang, Google; T. F. Abdelzaher,
%Current address) NASA Ames Research Center, Moffett Field, California 94035.
%\end{bottomstuff}

\maketitle

\include{sec/introduction}
\section{Prior Work}
Electricity disaggregation uses the electricity consumption level at the main entry into a building or house 
to infer whether a device inside the building is on or off. 
The features used include initial real power and reactive power~\cite{hart1992} from a dataset which is recorded in 
a low-frequency range. 
With advances in electrical meter technology and the availability of less expensive meters, 
more and more features are being extracted from the high-frequency data set and used for disaggregation, such as 
the transient state generated when a device turns on or off~\cite{shaw2000PhdThesis},
the raw current waveform~\cite{srinivasan2006neural}, the voltage waveform~\cite{lam2007novel}, 
the transform of the current waveform~\cite{chan2000harmonics}, 
and harmonics of non-linear devices~\cite{chan2000harmonics}. 
Even on-AC power features such as power line noises~\cite{patel2007flick}
are exploited jointly with AC power features like  
time of day, and device correlations~\cite{kim2011unsupervised}
in modern systems.
Increasingly,  
research is being focused on unsupervised learning and semi-supervised learning algorithms because  
these algorithms do not require the power consumption of 
each device,   
and the power of individual devices are very difficult to obtain. 
It is only in
 the last few years that 
unsupervised learning algorithms
have been used, including
hierarchical clustering~\cite{gonccalves2011unsupervised},
factorial hidden Markov models (FHMMs)~\cite{kim2011unsupervised},
additive factorial approximate MAPs (AFAMAP)~\cite{kolter2012aistat}, 
difference FHMMs~\cite{parson2012nonintrusive}, 
and motif mining~\cite{shao2013temporal}.
Semi-supervised learning 
algorithms~\cite{lam2007novel,johnson2012bayesian} have also
been proposed.
In this paper, we assume the number of devices 
and the number of power level states of each device 
are known. Hence, we formalize the disaggregation 
as a semi-supervised problem and 
provide solutions to the following three challenging problems.
\begin{enumerate}
\item Several devices may have the same real power, and
it is difficult to distinguish these devices using only the recorded aggregated
power time stamp.
\item Many devices may turn on or off at the same time.
\item Instead of having a discrete range of power
levels, there are devices whose power consumption levels
   vary gradually, e.g.,
  variable speed devices (VSD) and lights with dimmers.
Once their power usage is aggregated with that from other devices,
disaggregation becomes increasingly difficult.
\end{enumerate}
Since obtaining a low-frequency dataset is more practical in real buildings, 
we focus mainly on real power, which can be easily extracted 
from a low-frequency dataset.  
%Water use disaggregation has emerged in recent years, and so far 
%the applied algorithms are 
%limited to supervised learning algorithms~\cite{carboni2016contextualising}. 
%This chapter proposes water disaggregation as a semi-supervised learning algorithm 
%by presuming that we know the number of water use ends and the water usage level of each 
%water user end.  

%% prior work for occupancy prediction
Accurately predicting whether a home is occupied is a difficult task. 
People in the same home have different daily schedules; 
some go to work and others stay at home for a period of time.  
A great deal of research has been done to track the activities of people 
to infer the home occupancy. 
Researchers have made efforts to collect data by sensors, smart phones, 
the calendar, and weather information. 
Most of the approaches that model and predict occupancy primarily use sensor data to detect conditions 
such as room occupancy, use of electrical appliances, water usage, etc.
Several supervised learning approaches, such as kNN, neural networks, rule-based models, 
and Markov chain models have been used to model and predict building occupancy 
\cite{scott2011preheat,alrazgan2011learning,mahmoud2013behavioural,erickson2010occupancy,beltran2014optimal}.  
Using the kNN supervised learning algorithm and monitoring sensor data 
for a portion of the day, 
Scott et al. predict an entire day's occupancy in~\cite{scott2011preheat}. 
A neural network approach using a binary time series based on 
occupancy/unoccupancy along with exogenous input network (NARX) is 
proposed in \cite{mahmoud2013behavioural}. 
Mahmoud et al. tackle the problem by presenting a non-linear autoregressive 
model with an exogenous input (NARX) network. 
Several Markov chain models, like the blended Markov chain, 
closest distance Markov chain, 
and moving-window Markov chains are presented in \cite{erickson2010occupancy}. 
A mixture of multi-lag Markov chains was used to predict the occupancy of 
single-person offices \cite{manna2013learning}. 
In that work, the authors also compare their model with the Input Output Hidden Markov Model, 
First Order Markov Chain and the NARX neural network. 

A recent survey~\cite{kleiminger2014predicting} compares major occupancy 
predictions algorithms against the LDCC dataset~\cite{kiukkonen2010towards}, which was collected by 
GPS and other sensors. 
It shows that time-based presence probability~\cite{krumm2011learning} performs slightly better than the preheat kNN approach~\cite{scott2011preheat}. 
Since the preheat kNN approach~\cite{scott2011preheat} is more widely applicable,  
in that it can be used against both GPS and sensor datasets, 
we set it as a baseline method for comparison. 

%%%%%comment several paragraphs
\iffalse
These superseded learning approaches are classified into several categories. 
The first is on the probability density distribution of key events. 
\cite{tominaga2012unified} proposes that at a time,  a person goes out has a Bernoulli distribution. 
The second effective benchmark approach is kNN. 
kNN approach is employed in 
\cite{scott2011preheat} to predict the occupancy of the left day 
after knowing the occupancy in the partial day. 
It splits the whole day's time into 96 15-minutes intervals 
then to find the top-5 similar day in the training date. 
The average of these similarity is the predictive occupancy. 
The third is the pattern discovery by rule and neural network. 
A rule-based approach is proposed by
\cite{alrazgan2011learning}  for occupancy prediction under the frame work of Decision Guidance Query Language (DGQL). 
A variant of neural network has been proposed by 
\cite{mahmoud2010occupancy}. \cite{mahmoud2010occupancy} converts the data into binary occupancy/unoccupancy data in the first step. Then a model name non-linear autoregressive network with exogenous input (NARX) network is modeled for prediction. 
\cite{mahmoud2013behavioural} also uses binary time series with NARX network. 
The last are models related to Markov chains. 
Several Markov Chains have been compared in the paper of \cite{erickson2014occupancy}, including blended Markov Chain, closest distance Markov Chain, and the moving window Markov Chain with respect to modeling occupancy. 
\cite{erickson2010occupancy} uses moving-window markov chain for occupancy prediction. 
\cite{erickson2013poem} utilizes the markov chain model and blend markov chain model for prediction. 
\cite{beltran2014optimal} uses a blend-markov chain model for prediction. 
\cite{manna2013learning} uses mixture of multi-lag markov chains to predict the occupancy in single person offices. It compares with other previous approaches Input Output Hidden Markov Model, First order Markov Chain and NARX Neural Network. 

This paper contributes the follows:
1) formulate the problem as a temporal mining problem;
2) mine the activity patterns according to time and gap;
3) the occupancy prediction performance of this temporal mining approach works better than kNN for most cases.
\fi

\section{Multivariate Motif Mining to Disaggregation}
With the advent of modern sensor technologies, 
significant opportunities have emerged to help conserve energy in 
residential and commercial buildings. Moreover, the rapid \emph{urbanization} we are witnessing requires optimized energy distribution. 
Energy disaggregation attempts to 
separate the energy usage 
of each circuit or each electric device in a building 
using only aggregate electricity usage information from 
the whole house meter. 
Usually two-phase or three-phase electric power is 
connected to residential and commercial buildings. 
%Similarly, water disaggregation aims to discover each 
%water use end by only knowing the 
%hot and cold water usage from the whole house water meter.
We tackle energy disaggregation as a multiple-phase data disaggregation problem. 
The aim of this section is to identify electrical devices from 
two phases of aggregated data. 
Unlike previous work which disaggregate devices
from the sum of multiple phases, 
the time series information from each phase and the correlation of a device between/among phases 
are fully used.  
All of this information enables us to characterize more devices. 

%This work makes the following contributions in the field of disaggregation:
%\begin{enumerate}
%\item It can disaggregate aggregate data from multiple phases.
%\item It can separate the continuously variable loads which are mixed in electricity. 
%\item This approach can be used for both electricity disaggregation and water disaggregation.
%\end{enumerate}

\subsection{Disaggregation Formalism}
We propose a semi-supervised approach for disaggregation; i.e.
We assume that we know the on and off events for a short period of time for all devices, %or water end uses, 
and use that information to deduce the power levels, %or water usage, 
or to obtain the startup vectors of every device.

For our purposes, we define the disaggregation problem as follows:
Given $K$-phase aggregated power 
%or $K$ aggregated water consumption time series 
$Y_k=y_1^{(k)}, ..., y_T^{(k)}$, and a set of
power %or water related 
and contextual features, $f=f_1, ..., f_T$ over a period of time T, 
the problem is to estimate the disaggregated power %or water consumption 
of $M$ devices 
$\hat{X_m }= \hat{x}_{1}^{(m)}, ...\hat{x}_{t}^{(m)}, ... \hat{x}_{T}^{(m)}, m\in[1, M]$, 
such that a loss function of the sum of the power %or water consumption 
of the $M$
devices and the sum of the $K$ phases of aggregated power %or water consumption 
is minimized. 
\begin{equation}
\label{eq_powerObj}
\min_{\hat{x}_{t}^{(m)}} \{ \sum_{t=1}^T \mathscr{L}_t(\sum_{m=1}^M \hat{x}_{t}^{(m)}, \sum_{k=1}^Ky_t^{(k)}) \},
\end{equation}
where $\mathscr{L}_t$ is the loss function between 
the sum of $M$ estimated time series at $t$, 
and $y_t^{(k)}$ is the ground truth phase $k$ aggregated power %or water 
feature at time $t$. 
$\mathscr{L}$ is usually the $\mathscr{L}1$-norm $|\sum_{m=1}^M \hat{x}_{t}^{(m)} -\sum_{k=1}^K y_t^{(k)}|$
or the $\mathscr{L}2$-norm $(\sum_{m=1}^M \hat{x}_{t}^{(m)}-\sum_{k=1}^Ky_t^{(k)})^2$.

\subsection{Recursive Multivariate Piecewise Motif Mining}
To solve the problem of separating a multi-dimensional time series into several time series, 
I propose the approach of recursive multivariate piecewise motif mining. 
Motif mining has been well studied in previous work \cite{motif1} and \cite{motif2}. 
Multivariate or multidimensional motif mining is further extended in \cite{minnen2007improving} and \cite{tanaka2005discovery} and \cite{motifgoal}. 

Motif mining is applied to energy disaggregation in \cite{shao2013temporal}, 
in which discrete on/off events are exploited. 
This research enhances previous work by piecewise motif mining, 
where the on/off event is comprised of several consecutive data points, 
i.e. piecewise, other than individual discrete one.
Also, I use multivariate motif mining to make full use of two- or three-phase aggregated data. 

The framework of recursive multivariate piecewise motif mining to energy disaggregation is illustrated in Figure \ref{fig_multivariateMotifming}. 
The input includes multiple-phase aggregated data, such as two-phase data Mains1 and Mains2, 
and the power levels of each device. 
During the whole procedure, I recursively apply piecewise motif mining to two-phases and single phase diffs data.
The first step is to identify electrical devices which draw power from both phases.
Generally these devices consume a large amount of power, such as the water heater indicated by the blue line.
These devices draw equal power or disparate power from both phases synchronously. 
%draws equal power from both phases. 
%By comparing the diffs of these two phases, 
%we discover the same power changes during the on/off events. 
%With multivariate motif mining, 
%we can identify these large power consumption devices. 
Secondly, I remove the power consumption of the devices which draw power from both phases.
This action decreases the noise interference caused by large power consumption 
and increases the possibility to disaggregate more devices with low-power consumption. 
Then we apply piecewise motif mining to single-phase data to separate 
the devices that draw power only from that phase, 
such as the humidifier indicated by the green line. 
\input{multidisaggfig/multivariateMotifmining}
Generally, multivariate piecewise motif mining is divided into four steps, as shown in Figure \ref{fig_multivariatePiecewiseMotifMining}.
Step 1 is to search for piecewise events from the two-phase or three-phase data.
Step 2 is to encode events from multiple phases. 
Step 3 aims to mine frequent motifs from the encoded events list.
The last step targets to recover devices from mined motifs. 
\begin{figure}[h]
\centering
\includegraphics[width=0.7\textwidth]{multidisaggfig/multivariatePiecewiseMotifMining.pdf}
\caption{Multivariate Piecewise Motif Mining.}
\label{fig_multivariatePiecewiseMotifMining}
\end{figure}

\subsubsection{Piecewise Motif Mining}
Motif mining aims to uncover the repetitive patterns in time series data, and works best for discrete events. Piecewise motif mining is proposed 
for energy disaggregation to detect on/off events. 

\begin{definition}{\textbf{Piecewise Event}}
Given a time series diffs data $y_1, ...., y_{n'}$, where $\forall$ $|y_i| < \eta$. 
A piecewise event is the sum of these $n'$ number of diffs data, $e= \sum_{i=1}^{n'} y_i$. 
\end{definition}
Each piecewise event corresponds to an on/off event of an electric device. 
The value of $\eta$ is the noise range of each device, 
which is usually less than the 10\% of $|e|$.  

\textbf{Piecewise Events Search from Multiple Phases} \\
The majority of electrical devices which draw power from multiple phases consume larger amounts of power than electrical devices which connect to single phase. 
To disaggregate such a device, we need to 
discover specific on/off events features to separate them. 
%on/off events which are yielded by this single device from two-phase or three-phase aggregated data. 
Generally such an electrical device draws power from multiple phases synchronously 
and constructs a  pattern. 
Some devices may consume equal power from both phases all the time,  and so their power consumption patterns from both phases keep the same.
Other devices may show different power usage patterns when drawing power from two phases. 
%The power consumption from both phases may be exactly synchronized and keep the same all the time. 
%Either, the power consumption drawn from a single device differs much. 
\input{multidisaggalg/synchronizeEvents}
Algorithm~\ref{alg_synchronizeEvents} describes how synchronized events from two phases are revealed.  
This input include the two-phase aggregated diffs data and big power consumption threshold $\theta$. 
This threshold guarantees we only discover big power consumption devices. 
We review the phase-1 diffs data. 
If any absolute value $|y_i^{(1)}|$ is greater than $\theta$, 
both previous and posterior five diffs data points from time $i$ are checked. 
For these 10 points values, 
at each time $j$, if the difference between phase-1 $y_i^{(1)}$ and phase-2 $y_i^{(2)}$ is in the range of $0.2*|y_i^{(1)}|$,  
we assert that the diffs data points from these two phases are relatively the same and synchronized. 
The synchronization implies that 
these two identical amounts power consumption comes from a single device. 
Therefore we sum the synchronized power level diffs data and compute the power consumption at time $i$ as $e_i$.  
When $e_i>0$ that denotes an on event, and $e_i<0$ means an off event for a certain device.

Next we transfer these two-phase diffs data into 
an ordered on/off event list $e_1, ..., e_{n'}$,
then we apply motif mining to this events list. 
By matching the devices which consume power greater than $2*\theta$, 
we can separate all devices which draw equal amount of power from two phases.  
%Since we already know the power levels of each device, 
%we just choose those devices which include power levels bigger than $2*\theta$. 
%By applying motif mining, 
%we can separate all devices which consume two-phase power greater than $\theta$ equally. 
%For dataset Study10, we set $\theta=500W$. 
%This approach helps us to discover two devices waterHeater2 and heatingIndoor. 
%All of the on/off events of these two devices are found out. 

\subsubsection{Encoding Events From Multiple Phases}
After deleting all the synchronized events from phase-1 and phase-2, 
we apply multivariate piecewise motif mining to the remaining phase-1 and phase-2 diffs data, 
to detect devices which consume large amounts of power 
and draw power from two phases synchronously yet unequally. 
There are different power drawing patterns from these two phases.  
We encode these two-phase diffs data, which occur at the same time, as a new event $e$. 
Figure \ref{fig_eventEncoding} gives an example of how the events from two-phase circuits
are encoded. 
We extract an event which consumes power greater than $\theta$, 
then we check five more data points before and after it. 
The values of the 11 data points relevant to this event in Main1.diff are [0, 0, -18, 18, 1093, 1830, -196, -68, -37, -36, 0]. 
The concurrent events listed in Main2.diff are [0, 0, 0, 18, 9, 1946, 440, -51,-36,-36,0]. 
Since the events at the peak occur in the two phases as $(1830, 1946)$, 
and the difference of these two powers $1946-1830=116$ is in the $0.2*1830$ range, 
we consider that these two changes may come from a single device. 
When looking for insight into these two vectors, 
we observe that the sum of the changes of phase 1 is 2604W, and the sum of the changes of phase 2 is 2290W. 
They are in the same range, i.e. $2604*0.8 < 2290$. 
Therefore, we declare that the power changes from these two phases definitely come from a single device. 
We select two of these values and encode them as $e_{1'}=(1093, 9)$, $e_{2'}= (1830, 1946)$
\begin{figure}[h]
\centering
\includegraphics[width=0.5\textwidth]{multidisaggfig/synchronizeDifferentEventEncoding.pdf}
\caption{Encoding Events from Multiple Phases.}
\label{fig_eventEncoding}
\end{figure}

The piecewise events for this single device are $e= [e_{1'}, e_{2'}]$. 
Applying frequent motif mining, 
we separate this large power consumption device which draw power from two phases unequally. 





\include{sec/predictionApp}
\section{Disaggregation Results}
\subsection{Evaluation for Disaggregation}
%The evaluation tools has been discussed in prior work \cite{liang2010load}.
%
%To evaluate whether each devices is disaggregated or not,
%we need to evaluate two parts, one is how much energy
%is evaluated compared to the ground truth. Another is
%among those disaggregated energy, how much percentage
%are disaggregated right.
%
I use precision, recall and F-measure in our evaluation. The standard
definitions of these metrics are:
%\begin{equation}
$\textrm{precision}=\frac{TP}{TP+FP}$, 
%\end{equation}
%\begin{equation}
$\textrm{recall}=\frac{TP}{TP+FN}$,
%\end{equation}
%\begin{equation}
$\textrm{F-measure}=\frac{1}{\frac{1}{\textrm{precision}}+\frac{1}{\textrm{recall}}}$
%\end{equation}

We need to define the notions of true/false positives
and negatives in the context of disaggregation.

Let us suppose there is a ground truth time series $x$ with length T, 
and denote the corresponding disaggregated time series by $\hat{x}$.
For any time $t \in (0, T)$, there are two values: the
ground truth value of device $m$ is $x_t^{(m)}$ and the disaggregated value
$\hat{x}_t^{(m)}$. We define a parameter $\rho$ for the range of
true values $x_t^{(m)}$, and another parameter $\theta$
as the noise.
For any given measurement, 
there are four total power values or water usage values of device $m$ at
each point: true positive $TP^{(m)}$,  false negative $FN^{(m)}$,
true negative $TN^{(m)}$, and false positive $FP^{(m)}$.

\noindent
1. When $x_t^{(m)} > \theta$ and  $\hat{x}_t^{(m)}> \theta  $,
at this point the disaggregation is a true positive.
There are three situations in turn:

\noindent
1.1. When $  x_t^{(m)} \times (1-\rho) <  \hat{x}_t^{(m)} <  x_t^{(m)} \times (1+\rho)  $, then
\begin{eqnarray*}
 TP^{(m)} &=& \hat{x}_t^{(m)} \\
 FN^{(m)}&=&FP^{(m)} = TN^{(m)}=0
\end{eqnarray*}

\noindent
1.2. When $ \hat{x}_t^{(m)} < x_t^{(m)} \times (1-\rho)$ , then only
the disaggregated power or water usage is considered as true positive and
the power or water usage that is not disaggregated is regarded as a false negative:
\begin{eqnarray*}
TP^{(m)}&=&\hat{x}_t^{(m)} \\
FN^{(m)}&=&x_t^{(m)} - \hat{x}_t^{(m)} \\
FP^{(m)}&=&TN^{(m)}=0
\end{eqnarray*}
1.3 When $ \hat{x}_t^{(m)}>  x_t^{(m)} \times (1+\rho) $, then
the disaggregated power or water usage is a true positive, and those values
which are greater than the truth values are treated as false positive.
\begin{eqnarray*}
TP^{(m)}&=&\hat{x}_t^{(m)} \\
FP^{(m)}&=&\hat{x}_t^{(m)} - x_t^{(m)} \\
FN^{(m)}&=&TN^{(m)}=0
\end{eqnarray*}
2. When $x_t^{(m)} > \theta$ and  $\hat{x}_t^{(m)}< \theta  $,
at this point the disaggregation is a false positive.  Then,
\begin{eqnarray*}
FP^{(m)}&=&x_t^{(m)} \\
TP^{(m)}&=&FN^{(m)} =TN^{(m)}=0
\end{eqnarray*}
3. When $x_t^{(m)} < \theta$ and  $\hat{x}_t^{(m)} > \theta  $,
at this point the disaggregation is a false negative. Then,
\begin{eqnarray*}
FN^{(m)}&=&x_t^{(m)} \\
TP^{(m)}&=&FP^{(m)} = TN^{(m)}=0
\end{eqnarray*}
4. When $x_t^{(m)} < \theta$ and  $\hat{x}_t^{(m)} < \theta  $,
at this point the disaggregation is a true negative.  Then,
\begin{eqnarray*}
TP^{(m)}&=&FN^{(m)} = FP^{(m)} = TN^{(m)}=0
\end{eqnarray*}
%In practice, we use different
%$\rho$ and $\theta$ values depending on the
%dataset. For instance, considering the datasets described below,

In our experimental dataset, we set
$\theta=100$ and $\rho=0.2$. 
Although the maximal power consumption of all these devices is 11000W, 
we can still set $\theta < 11000 * 0.1$ because we apply multivariate piecewise motif mining recursively, 
so the devices which consume a large amount of power are deleted in the first few rounds. 
Therefore the power noise which is caused by the high-power electronic devices 
is greatly decreased. 

\subsection{Disaggregation Experiments}
We run experiments on the dataset Study10 from the University of Virginia on electricity disaggregation. 
This dataset collects data from 02/10/2014 to 02/21/2014 in a residential building. 
%the following paragraph is from UVa
Two individuals were asked to live in an instrumented home for around two weeks. 
To ensure the data consisted of the personal usage patterns of the participants, 
they were encouraged to live in and use the home as they normally would use their own. 
%The study had institutional review board (IRB) approval and each participant received a \$100 incentive. 
An eMonitor \cite{eMonitor} sensor was used to collect both mains data for testing and circuit-level information for ground truth. 
Additional data, such as the opening of appliance doors and the flicking of light switches, 
was collected to provide sub-circuit level ground truth information for events such as lights.
Both the two-phase aggregated data and each device's data are collected at intervals of 2-3 seconds.
In total, 25 devices were connected to two phases at the entry of the house. 
Five of these devices are seldomly operated; less than five times. 
Fourteen devices consume power less than 100W, and the majority of them are lights. 
The largest power consumption of these devices is 11000W by indoor heating. 
The noise caused by the heating device is large; greater than 100W. 
Therefore we focus on disaggregating the six major electronic devices 
with power levels greater than 100W. 
 
% the datasets from University of Virginia on both water and electricity disaggregation. 
%There are totally six data sets. The statistic of these six data set is listed in Table \ref{table_datasets}. 
%The recording interval of these datasets is 2 or 3 seconds. 
%A sensor is instrumented on each device to trace its ground truth operations. 
%\input{tab/datasets}

\subsubsection{Electricity Disaggregation}
We assume that we know the power levels of each device. 
If the power levels of each device are unknown, 
we can use the sum of two-phase aggregated data and the on/off events of the 
ground truth to extract them. 
%In order to extract the power level of each device, 
We set a window size $w=30s$ ahead and behind of the ground truth events to match 
the aggregated data.
If there is only one power change in the aggregated data during these 60 seconds, 
this power level change must come from an on/off event of an electrical device. 
Usually, it takes around 2-5 seconds for an electrical device to reach
a steady power level. 
The on and off events reflect different durations for a device to 
reach a steady state. 
Therefore, we measure the minimal duration of the on event and off event 
of each device. 
After we go over all the aggregated data and ground truth on/off events, 
we run a Gaussian mixture model to model the positive power changes and negative power changes
independently. The means and standard deviations correspond to  the on/off event of each device. 
The power levels, standard deviation, and on/off duration of each device of dataset study10 are listed in Table \ref{table_study10results}.
\input{multidisaggtab/study10results}

%By analyzing each device, we notice that sometimes the power levels and 
%on/off duration are insufficient to identify the electric devices. 
%For instance, when the device heatingIndoor starts alone, 
%it takes $2-5$ seconds to go to a steady state as shown in Figure \ref{fig_heatingDevices} (a).
%But when the combined device of heatingIndoor and heatingOutdoor 
%starts, the starting duration takes around $4-18$ seconds. 
%In Figure \ref{fig_patterns} (b), after this combined device 
%starts for15 seconds, the power levels changes 
%for 9 times then to a relatively stable state. 
%During this period of 15 seconds, the power changes very rapidly. 
%If we accumulate these power levels together, 
%it's not a fix number. Therefore, for this kind of device, 
%we can only compare the shape of the startup to decide the on events 
%from the aggregated data. 

We apply recursive multivariate piecewise motif mining to dataset Study10 
and compute the precision, recall and F-measure. 
Devices which draw power from both phases are separated first. 
They are heatingIndoor, waterheater and dryer.
\begin{figure*}[!t]
        \centering{
                \begin{tabular}{cc}
                \includegraphics[width=0.5\textwidth]{multidisaggfig/heatingIndoorPhase12On.png} &
                \includegraphics[width=0.5\textwidth]{multidisaggfig/heatingIndoorUp.png}\tabularnewline
                (a) & (b) \tabularnewline
                \includegraphics[width=0.5\textwidth]{multidisaggfig/heatingIndoorPhase12Off.png} &
                \includegraphics[width=0.5\textwidth]{multidisaggfig/heatingIndoorDown.png}\tabularnewline
                (c) & (d) \tabularnewline
                \end{tabular}
                }
        \caption{
        (a) On piecewise event and (c) Off piecewise event of heatingIndoor. heatingIndoor is disaggregated by motif mining the on event (b) and off event (d).}
        \label{fig_heatingIndoorResults}
\end{figure*}

Figure \ref{fig_heatingIndoorResults} (a) gives an example of 
an on event in the two-phase Mains1 and Mains2. 
Mains1.diff denotes the diffs data from Mains1 and Mains2.diff represents 
the diffs data from Mains2. 
Mains1.2.diff shows as blue when Mains1 and Mains2 share the similar power changes. 
We can see that 
the power consumption of a specific device jumps twice in two phases simultaneously.
The first time, both phases jump 2572W. After nine seconds, 
the power of both phases increases  2520W. 
The sum of these four changes is 10184W. 
Compared with the power levels of all devices, 
we speculate that these power changes are caused 
by the device heatingIndoor. 
%By applying multivariate piecewise motif mining, 
%we match the power consumption with $11000W$ and 
%categorize this on event into heating indoor. 
Figure \ref{fig_heatingIndoorResults} (b) shares the same snippet of time series as Figure \ref{fig_heatingIndoorResults} (a). 
The red line indicates that the on event of heatingIndoor is recognized.    
\begin{figure*}[!t]
        \centering{
                \begin{tabular}{cccc}
                \includegraphics[width=0.5\textwidth]{multidisaggfig/dryerPhase12OnDiff.png} &
                \includegraphics[width=0.5\textwidth]{multidisaggfig/waterHeaterPhase12OnDiff.png} \tabularnewline
                (a) & (b) \tabularnewline
                \includegraphics[width=0.5\textwidth]{multidisaggfig/heatingIndoorOutdoorPhase12On_1.png} \tabularnewline
                %\includegraphics[width=0.5\textwidth]{multidisaggfig/UpToiletFitted.png}\tabularnewline
                (c) \tabularnewline
                \end{tabular}
                }
        \caption{
        Disaggregating dryer and continuous variable load heatingOutdoor with multivariate motif mining. The on event of a device and the corresponding diffs in the two phases for (a) dryer, (b) water heater, (c) heatingOutdoor. 
  	%(d) Disaggregating the toilet water use end with dynamic time-warping subsequence search. This Y-axis is water flow rate in 10000*liter/minute.
        }
        \label{fig_dryerResults}
\end{figure*}

Similarly, the off event plunges twice in two seconds -2877W and -1759W in both phases, as shown in Figure \ref{fig_heatingIndoorResults} (c).
The sum of this off event is -9272W. 
After matching the power levels, we categorize it as the off event of heatingIndoor as indicated in Figure \ref{fig_heatingIndoorResults} (d). 

The dryer has the same power level as the waterheater at around 4800W. 
If we disaggregate these two devices from the sum of the two phases, 
it's difficult to distinguish them,  
but with multivariate piecewise motif mining, these two devices 
can be distinguished. 

%Figure \ref{fig_dryerResults} (a) and (b) shows the on event of dryer from sum of phase 1 and phase 2, 
%and these two phases separately. 
Figure \ref{fig_dryerResults} (a) and (b) are the diff data of the dryer and waterheater from the two-phase circuit. 
We can see that the waterheater draws power from Phase 1 and Phase 2 at the same time,  
but the dryer shows a different pattern. It draws power from Phase 1 at a lower power of 1093W, then jumps to 1830W; 
at the same time, it draws power from Phase 2 at the high level of 1946W immediately. 
We encode the power usage as shown in Figure \ref{fig_eventEncoding}, then apply motif mining to disaggregate them. 
 
After deleting the power consumption from both aggregated phases, 
we apply piecewise motif mining again to a single phase. 
We then discover the humidifier from Phase 1 
and the microwave from Phase 2. 
%Note that the power consumption of humidifier and microwave overlaps sometimes, 
%which makes it hard to separate them. 
%But they draw power from different phases separately. 
%Multivariate motif mining can separate them. 
When we only disaggregate the sum of Phase 1 and Phase 2, 
the precision recall result of the microwave and humidifier is not very accurate 
because sometimes their power consumptions are similar. 
However, using multivariate motif mining, we can separate them very clearly 
with good precision and recall. 
The precision and recall results for the data set Study10 are listed in Table \ref{table_study10results}.

Recursive multivariate motif mining is capable of disaggregating continuous variable loads. 
%\input{multidisaggfig/heatingOutdoorResults}
Figure \ref{fig_dryerResults} (c) shows the diff data of heatingOutdoor from the two phases. 
During this on event, its power levels change nine times, then continue at a relatively stable state.
%Figure \ref{fig_heatingOutdoorResults} (a) illustrates the startup of heatingOutdoor.
%as shown in Figure \ref{fig_heatingOutdoorResults} (b) and (c). 
By applying piecewise motif mining, 
we can successfully identify this as the heatingOutdoor device 
after matching its power level. 
%The disaggregated result is displayed in Figure \ref{fig_dryerResults} (d).
%Note that this continuous variable load startup snippet can be discovered by dynamic time warping 
%subsequence search as well if this device is on without the intervene of other devices.
If another device  $D$ which draws from Phase 1 or Phase 2 is turned on or off during this period, 
multivariate piece-wise motif mining can still identify this heatingOutdoor device. 
This is because $D$ only uses one phase's power; 
hence its power change is not counted in our piecewise event. 

\subsubsection{Water Disaggregation and Constraints}
Water usage displays different characteristics. 
The total water consumption is zero most of the time.
Whenever a water use end is operated, water is consumed intensively for a period of time. 
Then it will stay off for a much longer time. 
We observe that the operations of water use ends reflect a series of user behaviors. 
For instance, a person may use the toilet in the bathroom first, 
then wash hands in the sink and finally take a shower afterwards. 
%This series of events highly affect electricity usage as well. 
%When the person enters into the bathroom, the light in the bathroom is turned on. 
%After using the bathroom, the person leaves and turns the light off. 
%Moreover, we observe that the usage of toilet is usually accompanied by
%sink usage afterwards. 

%In this subsection, we apply the semi-supervised multivariate piecewise motif mining 
%to water data. 
Similar to electricity disaggregation, we use a period of aggregated water usage 
data to extract features  
and obtain the water flow rate level of each water use end.
Table \ref{table_resultStudy10Water} lists the water consumption rate for each device. 
For instance, taking a shower uses hot water at a flow rate between 0.1822 liter/min and 0.1986 liter/min. 
Let $\frac{\alpha}{10000}$ denotes this range of water flow rate.
The total hot and cold water consumption by  shower is 0.1904 liter/min. 
Therefore, the cold water flow rate caused by shower is $0.1904-\frac{\alpha}{10000}$ liter/min. 
Turning on the water for the shower takes around two seconds. 
%The disaggregation results are shown in Table \ref{table_resultStudy10Water}.
%\input{multidisaggtab/study10waterresults}
\begin{table}[h]
\renewcommand{\arraystretch}{1.3}
%\caption{Water Flow Rate Levels of Water End Uses.}
%\label{table_resultStudy10Water}
\tbl{Water Flow Rate Levels of Water End Uses and Disaggregation Results.\label{table_resultStudy10Water}}{
\centering
\small
\setlength\tabcolsep{2pt}
\begin{tabular}{|c|c|c|c|c|c|c|}
\hline
\multirow{2}{*}{Device} & \multirow{2}{*}{Hot water} & \multirow{2}{*}{Cold water} & \multirow{2}{*}{Duration}  \\
           &  (liter/min*10000) & (l/min*10000)  & (second)\\
\hline
\hline
Shower & $\alpha \in (1822, 1986)$ & $1904-\alpha$  & on: 2\\
\hline
Washing Machine & $\alpha \in (1988, 2276)$  & $2132-\alpha$ & on: 5\\
\hline
DownToilet & 0 & (1270, 1400) & whole: 50\\
\hline
UpToilet & 0 & (1480, 1700) & whole: 50\\
%\hline
%KitchenSink &  $ \alpha \in (0, 57) $ & 57- $\alpha$ & 2& & & \\
%\hline
%UpSink & 34 & 160 & 2& & & \\
%\hline
%DownSink & 57  & 80 & 2& & & \\
%\hline
%Dish Washer & 34 & -11 & 2& & & \\
\hline
\end{tabular}
}
\end{table}

After these calculations, we apply a multivariate piecewise motif mining approach to 
water disaggregation. 
For the shower and washing machine, 
the total flow rate of hot and cold water is high,  nearly 0.2 liter/minute. 
Therefore by only searching the 
total hot and cold water flow rate, we can identify these two devices. 
The event of shower usage usually lasts for more than one minute, 
but the washing machine uses water for less than one minute, 
repeating six to nine times. 
Both the shower and washing machine use 
hot water and cold water. 
However, the washing machine uses hot water for only the 
first one or two times. For the rest of its cycle, 
only cold water is used. 
Whenever the washing machines starts, 
the power consumption starts as well. 

%\input{multidisaggfig/waterDisaggResults}
%Figure \ref{fig_waterDisaggResults} (a) and (b) display the water usage of shower and washing machine. 
Applying piecewise motif mining to the water usage lets us disaggregate the
shower and washing machine. %as in Figure \ref{fig_waterDisaggResults} (a) and (b).
The precision, recall and F-measure for the shower disaggregation are 
0.999, 0.972, and 0.986,  
and the precision, recall and F-measure for the washing machine disaggregation are 
0.997, 0.969, and 0.983. 
However, with a variable water flow rate, 
piecewise motif mining has limitations in handling water use ends such as the toilet.
Therefore we use the dynamic time warping subsequence~\cite{rakthanmanon2012searching} search as a complementary to discover these water use ends.  
For the two toilets, we apply dynamic time warping to match the time series.
The water usage results of one toilet, UpToilet, is shown in Figure \ref{fig_dryerResults} (d). 

%There are totally three sinks in the house, namely up sink, down sink and kitchen sink. 
%People may use sinks with only hot water or cold water, or both hot and cold water. 
%The water usage may be large or small. 
%In this case, it is hard to distinguish these three sinks. 
%However, by observing the water usage of up sink and down sink. 
%We find that there's correlation between the down sink and the down toilet, 
%up sink and up toilet. 
%Figure \ref{fig_ToiletSinkCorrelation} (a) and (b) show that the toilet and sink start almost at the same time, 
%or end at the same time. 
We can see that multivariate piecewise motif mining is capable of 
disaggregating water use ends which have sharp on/off water flow rates. 
However, it has limitations in dealing with water use ends with irregular water use patterns, 
such as toilets and sinks. 
Since the water usage of toilets is relatively fixed if used alone, 
some toilet water usages can be disaggregated by using the dynamic time warping subsequence search 
which was researched in \cite{nguyen2013development}. 


\include{sec/predictionResults}
\input{sec/conclusion.tex}


%\input{multidisaggsec/introduction.tex}
%\section{Prior Work}
Electricity disaggregation uses the electricity consumption level at the main entry into a building or house 
to infer whether a device inside the building is on or off. 
The features used include initial real power and reactive power~\cite{hart1992} from a dataset which is recorded in 
a low-frequency range. 
With advances in electrical meter technology and the availability of less expensive meters, 
more and more features are being extracted from the high-frequency data set and used for disaggregation, such as 
the transient state generated when a device turns on or off~\cite{shaw2000PhdThesis},
the raw current waveform~\cite{srinivasan2006neural}, the voltage waveform~\cite{lam2007novel}, 
the transform of the current waveform~\cite{chan2000harmonics}, 
and harmonics of non-linear devices~\cite{chan2000harmonics}. 
Even on-AC power features such as power line noises~\cite{patel2007flick}
are exploited jointly with AC power features like  
time of day, and device correlations~\cite{kim2011unsupervised}
in modern systems.
Increasingly,  
research is being focused on unsupervised learning and semi-supervised learning algorithms because  
these algorithms do not require the power consumption of 
each device,   
and the power of individual devices are very difficult to obtain. 
It is only in
 the last few years that 
unsupervised learning algorithms
have been used, including
hierarchical clustering~\cite{gonccalves2011unsupervised},
factorial hidden Markov models (FHMMs)~\cite{kim2011unsupervised},
additive factorial approximate MAPs (AFAMAP)~\cite{kolter2012aistat}, 
difference FHMMs~\cite{parson2012nonintrusive}, 
and motif mining~\cite{shao2013temporal}.
Semi-supervised learning 
algorithms~\cite{lam2007novel,johnson2012bayesian} have also
been proposed.
In this paper, we assume the number of devices 
and the number of power level states of each device 
are known. Hence, we formalize the disaggregation 
as a semi-supervised problem and 
provide solutions to the following three challenging problems.
\begin{enumerate}
\item Several devices may have the same real power, and
it is difficult to distinguish these devices using only the recorded aggregated
power time stamp.
\item Many devices may turn on or off at the same time.
\item Instead of having a discrete range of power
levels, there are devices whose power consumption levels
   vary gradually, e.g.,
  variable speed devices (VSD) and lights with dimmers.
Once their power usage is aggregated with that from other devices,
disaggregation becomes increasingly difficult.
\end{enumerate}
Since obtaining a low-frequency dataset is more practical in real buildings, 
we focus mainly on real power, which can be easily extracted 
from a low-frequency dataset.  
%Water use disaggregation has emerged in recent years, and so far 
%the applied algorithms are 
%limited to supervised learning algorithms~\cite{carboni2016contextualising}. 
%This chapter proposes water disaggregation as a semi-supervised learning algorithm 
%by presuming that we know the number of water use ends and the water usage level of each 
%water user end.  

%% prior work for occupancy prediction
Accurately predicting whether a home is occupied is a difficult task. 
People in the same home have different daily schedules; 
some go to work and others stay at home for a period of time.  
A great deal of research has been done to track the activities of people 
to infer the home occupancy. 
Researchers have made efforts to collect data by sensors, smart phones, 
the calendar, and weather information. 
Most of the approaches that model and predict occupancy primarily use sensor data to detect conditions 
such as room occupancy, use of electrical appliances, water usage, etc.
Several supervised learning approaches, such as kNN, neural networks, rule-based models, 
and Markov chain models have been used to model and predict building occupancy 
\cite{scott2011preheat,alrazgan2011learning,mahmoud2013behavioural,erickson2010occupancy,beltran2014optimal}.  
Using the kNN supervised learning algorithm and monitoring sensor data 
for a portion of the day, 
Scott et al. predict an entire day's occupancy in~\cite{scott2011preheat}. 
A neural network approach using a binary time series based on 
occupancy/unoccupancy along with exogenous input network (NARX) is 
proposed in \cite{mahmoud2013behavioural}. 
Mahmoud et al. tackle the problem by presenting a non-linear autoregressive 
model with an exogenous input (NARX) network. 
Several Markov chain models, like the blended Markov chain, 
closest distance Markov chain, 
and moving-window Markov chains are presented in \cite{erickson2010occupancy}. 
A mixture of multi-lag Markov chains was used to predict the occupancy of 
single-person offices \cite{manna2013learning}. 
In that work, the authors also compare their model with the Input Output Hidden Markov Model, 
First Order Markov Chain and the NARX neural network. 

A recent survey~\cite{kleiminger2014predicting} compares major occupancy 
predictions algorithms against the LDCC dataset~\cite{kiukkonen2010towards}, which was collected by 
GPS and other sensors. 
It shows that time-based presence probability~\cite{krumm2011learning} performs slightly better than the preheat kNN approach~\cite{scott2011preheat}. 
Since the preheat kNN approach~\cite{scott2011preheat} is more widely applicable,  
in that it can be used against both GPS and sensor datasets, 
we set it as a baseline method for comparison. 

%%%%%comment several paragraphs
\iffalse
These superseded learning approaches are classified into several categories. 
The first is on the probability density distribution of key events. 
\cite{tominaga2012unified} proposes that at a time,  a person goes out has a Bernoulli distribution. 
The second effective benchmark approach is kNN. 
kNN approach is employed in 
\cite{scott2011preheat} to predict the occupancy of the left day 
after knowing the occupancy in the partial day. 
It splits the whole day's time into 96 15-minutes intervals 
then to find the top-5 similar day in the training date. 
The average of these similarity is the predictive occupancy. 
The third is the pattern discovery by rule and neural network. 
A rule-based approach is proposed by
\cite{alrazgan2011learning}  for occupancy prediction under the frame work of Decision Guidance Query Language (DGQL). 
A variant of neural network has been proposed by 
\cite{mahmoud2010occupancy}. \cite{mahmoud2010occupancy} converts the data into binary occupancy/unoccupancy data in the first step. Then a model name non-linear autoregressive network with exogenous input (NARX) network is modeled for prediction. 
\cite{mahmoud2013behavioural} also uses binary time series with NARX network. 
The last are models related to Markov chains. 
Several Markov Chains have been compared in the paper of \cite{erickson2014occupancy}, including blended Markov Chain, closest distance Markov Chain, and the moving window Markov Chain with respect to modeling occupancy. 
\cite{erickson2010occupancy} uses moving-window markov chain for occupancy prediction. 
\cite{erickson2013poem} utilizes the markov chain model and blend markov chain model for prediction. 
\cite{beltran2014optimal} uses a blend-markov chain model for prediction. 
\cite{manna2013learning} uses mixture of multi-lag markov chains to predict the occupancy in single person offices. It compares with other previous approaches Input Output Hidden Markov Model, First order Markov Chain and NARX Neural Network. 

This paper contributes the follows:
1) formulate the problem as a temporal mining problem;
2) mine the activity patterns according to time and gap;
3) the occupancy prediction performance of this temporal mining approach works better than kNN for most cases.
\fi

%\input{multidisaggsec/formalism.tex}
%\input{multidisaggsec/approach.tex}
%\input{multidisaggsec/evaluation.tex}
%\input{multidisaggsec/electricityExp.tex}
%\input{multidisaggsec/waterExp.tex}
%\input{multidisaggsec/conclusion.tex} 

%\input{adlsec/abstract.tex}
%\input{adlsec/introduction.tex}
%\input{adlsec/related.tex}
%\input{adlsec/formulation.tex}
%\input{adlsec/algorithm.tex}
%\input{adlsec/result.tex}
%\input{adlsec/conclusion.tex}
%\input{adlsec/appendix.tex}

%\input{sec/conclusion.tex}

% Algorithm
\iffalse
\begin{algorithm}[t]
\SetAlgoNoLine
\KwIn{Node $\alpha$'s ID ($ID_{\alpha}$), and node $\alpha$'s
neighbors' IDs within two communication hops.}
\KwOut{The frequency number ($FreNum_{\alpha}$) node $\alpha$ gets assigned.}
$index$ = 0; $FreNum_{\alpha}$ = -1\;
\Repeat{$FreNum_{\alpha} > -1$}{
        $Rnd_{\alpha}$ = Random($ID_{\alpha}$, $index$)\;
        $Found$ = $TRUE$\;
        \For{each node $\beta$ in $\alpha$'s two communication hops
    }{
      $Rnd_{\beta}$ = Random($ID_{\beta}$, $index$)\;
      \If{($Rnd_{\alpha} < Rnd_{\beta}$) \text{or} ($Rnd_{\alpha}$ ==
          $Rnd_{\beta}$ \text{and} $ID_{\alpha} < ID_{\beta}$)\;
      }{
        $Found$ = $FALSE$; break\;
      }
        }
     \eIf{$Found$}{
           $FreNum_{\alpha}$ = $index$\;
         }{
           $index$ ++\;
     }
      }
\caption{Frequency Number Computation}
\label{alg:one}
\end{algorithm}
\fi

% Table
\iffalse
\begin{table}%
\tbl{Simulation Configuration\label{tab:one}}{%
\begin{tabular}{|l|l|}
\hline
TERRAIN{$^a$}   & (200m$\times$200m) Square\\\hline
Node Number     & 289\\\hline
Node Placement  & Uniform\\\hline
Application     & Many-to-Many/Gossip CBR Streams\\\hline
Payload Size    & 32 bytes\\\hline
Routing Layer   & GF\\\hline
MAC Layer       & CSMA/MMSN\\\hline
Radio Layer     & RADIO-ACCNOISE\\\hline
Radio Bandwidth & 250Kbps\\\hline
Radio Range     & 20m--45m\\\hline
\end{tabular}}
\begin{tabnote}%
\Note{Source:}{This is a table
sourcenote. This is a table sourcenote. This is a table
sourcenote.}
\vskip2pt
\Note{Note:}{This is a table footnote.}
\tabnoteentry{$^a$}{This is a table footnote. This is a
table footnote. This is a table footnote.}
\end{tabnote}%
\end{table}%
\fi

% Acknowledgments
%\begin{acks}
%The authors would like to thank Dr. Maura Turolla of Telecom
%Italia for providing specifications about the application scenario.
%\end{acks}

% Bibliography
\bibliographystyle{ACM-Reference-Format-Journals}
\bibliography{references}

% History dates
%\received{February 2007}{March 2009}{June 2009}

\end{document}



